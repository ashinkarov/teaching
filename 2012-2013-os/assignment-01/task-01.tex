\documentclass[a4paper,10pt]{scrartcl}

\usepackage[english]{babel}

% Customizable sections
\usepackage{sectsty}

\usepackage{fancyhdr}
\fancyhead[R]{} %\includegraphics[height=2cm]{hw.pdf}}
\fancyhead[L]{{\sc F29OC Operating Systems and Concurency }}
 
\fancyfoot[L]{Hand-out: 15/01/2013}
\fancyfoot[R]{Hand-in:  29/01/2013}
\fancyfoot[C]{}
\pagestyle{fancy}

\renewcommand{\headrulewidth}{0pt}

\usepackage{graphicx}



\title{ 
    %{\sc F29OC Operating Systems and Concurency } \\
    Task 01
}
\date{}

\begin{document}
    \maketitle
    \thispagestyle{fancy}

    \section*{General requirements}
    Please submit the assignment as a \verb|*.zip| archive, where every problem is stored in
    its own subdirectory.  All the code must be well-commented and must come with a Makefile 
    which builds a program.  Notes and explanations may come in a separate file in the problem's
    directory.  For notes and explanations use text or pdf format.

    \section{Problem 1}
    Find out what the following terms mean when speaking about UNIX processes:
    \textit{zombie}, \textit{orphan}, \textit{daemon}.  
    Answer the following questions providing reasonable explanations:
    \begin{enumerate}
        \item What happens when a child kills its parent?
        \item Can a zombie do so, or it is too dead for that?
        \item Can a daemon become a zombie?
        \item How long can an orphan live on it's own, if at all?
    \end{enumerate}


    \section{Problem 2.1}
    Write a program in C which reports local time in the system log every 10 seconds.  
    
    Use \verb|time.h| functions to get local time and convert it to string, use \verb|syslog.h|
    to handle logging facilities.
    
    \section{Problem 2.2}
    Make sure that the time-reporting program from the previous problem keeps running after its
    parent or grandparents terminate.  \textit{Hint:} learn about daemons in UNIX.


    \section{Problem 3.1}
    Write a program in C that executes the \verb|ls| command, mirroring command line arguments,
    reads the output and outputs it on \textit{stdout} only if the number of lines is smaller
    than \textit{25}.  Otherwise the program should state: ``too many files''.

    Use the \verb|fork| function to cerate a subprocess; use the \verb|execv| family of functions
    to launch \verb|ls| and use pipes to read the output produced by the \verb|ls| command.  
    Command line arguments passed to the \verb|main| function via its second argument (\textit{argv})
    could be reused within \verb|execv| function call.  The following C code loads the content of
    the \textit{fd} file descriptor into memory.
{\small
\begin{verbatim}
        char* buf = NULL;
        char* bufp;
        size_t bufsz, cursz, curpos;
        ssize_t ssz;
        struct stat st;

        /* Assuming that FD is a file descriptor opened for
           reading, get file system I/O blocksize into BUFSZ.  */
        if (fstat (fd, &st) >= 0) {
                bufsz = (size_t) st.st_blksize;
        } else {
                /* Handle error.  */
        }

        /* Allocate buffer of size BUFSZ.  */
        buf = (char *) malloc (bufsz);
        curpos = 0;
        cursz = bufsz;

        /* Block read FD, storing data into BUF.  */
        while ((ssz = read (fd, buf + curpos, bufsz)) > 0) {
                curpos += ssz;
                cursz = curpos + bufsz;
                if (NULL == (bufp = (char *) realloc (buf, cursz))) {
                        /* Handle error.  */
                }
                buf = bufp;
        }

        /* Zero-terminate BUF.  */
        buf[curpos] = 0;
\end{verbatim}
}

    \section{Problem 3.2}
    Modify the previous task and pipe the output of the \verb|ls| command through 
    \verb|less| command if the output is longer than \textit{25} lines.

\end{document}

